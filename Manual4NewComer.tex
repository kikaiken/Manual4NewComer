\documentclass[a4paper,titlepage]{ujarticle}
\usepackage[top=20truemm,bottom=20truemm,left=10truemm,right=10truemm]{geometry}
\usepackage{here}
\usepackage{color}
\usepackage{colortbl}
\usepackage{graphicx}

\title{京都大学 機械研究会 新入生教本}
\date{\today}
\author{松岡 航太郎\\29年度機械研究会同期一同}

\begin{document}
\maketitle
\section{序文}
\section{抵抗}
回路の基礎中の基礎。電流の制限、分圧やプルアップなど、様々な用途に用いる。この節では用途ではなく抵抗選定のための知識に主眼を置く。
\subsection{カラーコード}
抵抗には、カラーコードと呼ばれる色のついた帯が印刷されている。必ずしも暗記する必要はない(暗記しなくても済むように整理しておく方が有意義である)が、よく使う抵抗値についてはばらまく可能性も高いので覚えておくことが好ましい。

各色の意味は以下の通りである。

\begin{table}[H]
	\begin{tabular}{|cc|c|c|c|}\hline
		色&&数値&乗率&精度\\ \hline
		無&&N/A&N/A&$\pm$20\%\\ \hline
		桃&\cellcolor[rgb]{1,0.412,0.706}{ }&N/A&$10^{-3}$&N/A\\ \hline
		銀&\cellcolor[rgb]{0.753,0.753,0.753}{ }&N/A&$10^{-2}$&$\pm$10\%\\ \hline
		金&\cellcolor[rgb]{0.812,0.71,0.231}{ }&N/A&$10^{-1}$&$\pm$5\%\\ \hline
		黒&\cellcolor{black}{ }&0&1&N/A\\ \hline
		茶&\cellcolor[rgb]{0.647059,0.164706,0.164706}{ }&1&10&$\pm$1\%\\ \hline
		赤&\cellcolor{red}{ }&2&$10^2$&$\pm$2\%\\ \hline
		橙&\cellcolor[rgb]{0.894118,0.368627,0}{ }&3&$10^3$&$\pm$3\% \\ \hline
		黄&\cellcolor{yellow}{ }&4&$10^4$&$\pm$5\%\\ \hline
		緑&\cellcolor{green}{ }&5&$10^5$&$\pm$0.5\%\\ \hline
		青&\cellcolor{blue}{ }&6&$10^6$&$\pm$0.25\%\\ \hline
		紫&\cellcolor[rgb]{0.58,0,0.827}{ }&7&$10^7$&$\pm$0.1\%\\ \hline
		灰&\cellcolor[rgb]{0.627,0.627,0.627}{ }&8&$10^8$&$\pm$0.05\%\\ \hline
		白&\cellcolor[rgb]{1,1,1}{ }&9&$10^9$&$\pm$0.025\%\\ \hline
	\end{tabular}
\end{table}
\subsection{種類}
一口に抵抗といっても、材質や用途によって区別される。ここではそれらの抵抗の種類について述べる。
\subsubsection{カーボン抵抗}
最も一般的な抵抗。精度は5\%程度で温度特性も悪いが安価であるためよく使われる。1/4W程度の定格電力であることが多い。
\subsubsection{金属皮膜抵抗}
精度は1\%以下であることが多く、温度特性もよいが、カーボン抵抗よりは高価である。分圧する場合などに精度が欲しいときは使うことになるだろう。1/8Wなど、カーボン抵抗よりも定格電力が低い傾向がある。
\subsubsection{セメント抵抗}
W単位の大電力を流したい場合に使用する。
\subsubsection{チップ抵抗}
面実装をするための抵抗のことを示す。
\subsubsection{シャント抵抗}
シャント(shunt)は英語では並列接続のことを意味するが、シャント抵抗といった場合は電流計に用いるための低抵抗、高精度、高定格電力な抵抗のことを言う。
\end{document}