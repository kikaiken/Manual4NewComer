\documentclass[a4paper,titlepage]{ujarticle}
\usepackage[top=20truemm,bottom=20truemm,left=10truemm,right=10truemm]{geometry}
\usepackage{here}
\usepackage{color}
\usepackage{colortbl}
\usepackage{graphicx}

\title{京都大学 機械研究会 新入生教本}
\date{\today}
\author{松岡 航太郎\\29年度機械研究会同期一同}

\begin{document}
\maketitle
\section{序文}
\subsection{抵抗}
回路の基礎中の基礎。電流の制限、分圧やプルアップなど、様々な用途に用いる。この節では用途ではなく抵抗選定のための知識に主眼を置く。
\subsubsection{カラーコード}
抵抗には、カラーコードと呼ばれる色のついた帯が印刷されている。必ずしも暗記する必要はない(暗記しなくても済むように整理しておく方が有意義である)が、よく使う抵抗値についてはばらまく可能性も高いので覚えておくことが好ましい。

各色の意味は以下の通りである。

\begin{table}[H]
	\begin{tabular}{|cc|c|c|c|}\hline
		色&&数値&乗率&精度\\ \hline
		無&&N/A&N/A&$\pm$20\%\\ \hline
		桃&\cellcolor[rgb]{1,0.412,0.706}{ }&N/A&$10^{-3}$&N/A\\ \hline
		銀&\cellcolor[rgb]{0.753,0.753,0.753}{ }&N/A&$10^{-2}$&$\pm$10\%\\ \hline
		金&\cellcolor[rgb]{0.812,0.71,0.231}{ }&N/A&$10^{-1}$&$\pm$5\%\\ \hline
		黒&\cellcolor{black}{ }&0&1&N/A\\ \hline
		茶&\cellcolor[rgb]{0.647059,0.164706,0.164706}{ }&1&10&$\pm$1\%\\ \hline
		赤&\cellcolor{red}{ }&2&$10^2$&$\pm$2\%\\ \hline
		橙&\cellcolor[rgb]{0.894118,0.368627,0}{ }&3&$10^3$&$\pm$3\% \\ \hline
		黄&\cellcolor{yellow}{ }&4&$10^4$&$\pm$5\%\\ \hline
		緑&\cellcolor{green}{ }&5&$10^5$&$\pm$0.5\%\\ \hline
		青&\cellcolor{blue}{ }&6&$10^6$&$\pm$0.25\%\\ \hline
		紫&\cellcolor[rgb]{0.58,0,0.827}{ }&7&$10^7$&$\pm$0.1\%\\ \hline
		灰&\cellcolor[rgb]{0.627,0.627,0.627}{ }&8&$10^8$&$\pm$0.05\%\\ \hline
		白&\cellcolor[rgb]{1,1,1}{ }&9&$10^9$&$\pm$0.025\%\\ \hline
	\end{tabular}
\end{table}
\subsection{種類}
一口に抵抗といっても、材質や用途によって区別される。ここではそれらの抵抗の種類について述べる。
\subsubsection{カーボン抵抗}
最も一般的な抵抗。精度は5\%程度で温度特性も悪いが安価であるためよく使われる。1/4W程度の定格電力であることが多い。
\subsubsection{金属皮膜抵抗}
精度は1\%以下であることが多く、温度特性もよいが、カーボン抵抗よりは高価である。分圧する場合などに精度が欲しいときは使うことになるだろう。1/8Wなど、カーボン抵抗よりも定格電力が低い傾向がある。
\subsubsection{セメント抵抗}
W単位の大電力を流したい場合に使用する。
\subsubsection{チップ抵抗}
面実装をするための抵抗のことを示す。
\subsubsection{シャント抵抗}
シャント(shunt)は英語では並列接続のことを意味するが、シャント抵抗といった場合は電流計に用いるための低抵抗、高精度、高定格電力な抵抗のことを言う。
\section{コンデンサ}
コンデンサは短期的に電気をためることができる素子である。(英語ではキャパシタということが多い)
\subsection{用途}
\subsubsection{パスコン}
少なくとも機械研究会において、最も多く用いる用途はパスコンであろう。パスコンとは、ICやモーターのできる限り近くの電源線とグラウンドの間にコンデンサを設置することで、電源線に伝わる高周波のノイズをグラウンドに落とすためのコンデンサのことである。

これはコンデンサのインピーダンスが1/jwCであらわされることから、十分に高周波なノイズに対しては短絡しているかのようにふるまうことに基づいている。
\subsubsection{電源安定化}
急激な消費電力の変化がある場合には、大容量のコンデンサを設置しておくことで、電源が安定するようにする場合がある。
\subsubsection{電源}
電気二重層コンデンサ(スーパーキャパシタ)などは容量に飽かせて電源として使うこともある。
\subsection{種類}
\subsubsection{アルミ電解コンデンサ}
最も一般的なコンデンサ。電子回路に円筒状の物体があったら大体こいつ。低価格大容量高耐圧だが極性があり、逆電圧をかけると劣化する。

昔は多くのアルミ電解コンデンサが電解液を用いていたために、劣化すると膨らんでいた。(俗に妊娠といわれた)現在は電解質が固体のものも出回っている。見分け方は円筒の上面に十字の切れ込みがあるか否か。電解液の場合には膨らむことができるように十字の切れ込みが入っている。
\subsubsection{セラミックコンデンサ}
アルミ電解コンデンサに劣らないぐらい一般的なコンデンサ。機械研究会の活動範囲ではこちらの方が使用機会が多いかもしれない。

高周波特性がアルミ電解コンデンサより良く安価で、小型化も容易であることからパスコンとして使われることが多い。容量が小さいものしかないことが難点だったが、積層セラミックコンデンサの登場でその状況も変わりつつあり、セラミックコンデンサオンリーで設計される場合も増えている。
\subsubsection{タンタルコンデンサ}
小型かつ大容量で周波数特性もよいことから、オーディオ機器などによく使われるが、高価で極性があり耐圧も低いことが多い。積層セラミックに地位を脅かされつつあるように見える。
\section{トランジスタ}
トランジスタは端的に言えばシリコンで作られた電気的に操作可能なスイッチである。
\subsection{用途}
\subsubsection{スイッチ}
最もよくある使い道。単純に機械的なスイッチの代わりに用いることでマイコンから操作可能なスイッチとすることができる。
\subsubsection{増幅}
トランジスタは現実に存在する素子であるから、理想的なスイッチと違ってオンとオフの間の状態が存在する。その部分を利用することで、小さな入力信号を大きな出力信号として増幅することができる。

主にオーディオ機器などのアナログ回路で利用される。
\subsection{種類}
\subsubsection{バイポーラトランジスタ}
電流によって制御されるトランジスタ。正孔と電子との両方が動作原理にかかわるのでバイポーラと呼ばれる。歴史的には初期に実用化されたトランジスタであり、かつてはトランジスタといえばこれのことを指していた。
\subsubsection{FET}
電圧によって制御されるトランジスタ。正孔と電子のいずれか片方だけが動作原理にかかわるため、ユニポーラトランジスタとも呼ばれる。電界効果トランジスタという名前もある。MOSFETという名称もよく聞くが、MOSは金属酸化物半導体の意であり、使用している材質に起因する名称であるが、近年のFETはほぼすべてMOSFETであり、近年トランジスタといった場合はほぼMOSFETのことである。

パワーMOSFETと呼ばれる、大電力用に設計されたMOSFETも存在する。
\subsubsection{IGBT}
FETとバイポーラトランジスタを組み合わせたような特性を持つ素子である。絶縁ゲートバイポーラトランジスタとも呼ばれる。大電力を高速でスイッチングすることができる。パワーMOSFETとは用途の近さから競争関係にある場合がある。
\section{バッテリー}
機械研究会では電源として商用電源を使わない場合が多いため、バッテリーに関する知識は重要である。
\subsection{用途}
\subsubsection{主電源}
特に説明の必要はないだろう。
\subsubsection{バックアップ電源}
主電源が商用電源など、コードが抜けたりブレーカーが落ちたりして突然寸断したとしても、主電源が復旧するまでの間機器を動かすための電源をバックアップ電源という。

そのような用途で使われるバッテリーは大型の場合UPSなどと呼ばれる。
\subsection{種類}
\subsubsection{単三乾電池}
最も安全で安価な選択肢。交渉電圧は一本あたり1.5V、容量は様々である。海外ではAAバッテリーともいう。主にアルカリとマンガンの二種類があり、アルカリが大電力向け、マンガンが小電力向けなどといわれているが、とりあえずアルカリにしておけば問題はないだろう。

単三電池で動くなら、単三にしておいた方がいい。もちろん、容量が問題なら単一、サイズが問題なら単四あたりを使ってもよい。
\subsubsection{ボタン電池}
容量が小さく小電力向けではあるが、サイズが小さいため小型機器の主電源やバックアップ電源として使われる。
\subsubsection{9V形電池}
角型をした電池で、容量は小さいが9Vという比較的大きな電圧を取り出すことができる。中身としては直列にボタン電池を重ねたようなものが多い。
\subsubsection{LiPoバッテリー}
最も危険だが最も高性能なバッテリー。
\subsubsection{LiFeバッテリー}
\subsubsection{NiMHバッテリー}
\subsubsection{鉛蓄電池}
\end{document}